\input{./input/PREAMBLE}
%
\include{./include/CAPA}
%
\tableofcontents
%
\appendix
%
\pagestyle{plain}%plain headings empty
%%%%%%%%%%%%%%%%%%%%%%%%%%%%%%%%%%%%%%%%%%%%
\newpage
\label{Resumo}
\begin{abstract}
Este trabalho consiste no estudo de Estatística das Entregas Expresso em  duas regiões \textbf{A} e \textbf{B}, as variaveis em estudo é o tempo de demora das entregas e a variavel de numero de encomendas entregues num determinado unidade de tempo [u.t.]. Nestas situações foram retiradas 120 e 90 amostras nas duas regiões respectivamente.\\
A primeira é uma distribuição continua, o tempo, e a segunda uma distribuição discreta.
\\
As materias abordadas vai ser \textbf{Amostragem}, \textbf{Estimação de parâmetros} e \textbf{Testes de Hipóteses}\\
\end{abstract}
%
\newpage
\section{Introdução}\label{Introdução}
%
As variáveis consideradas são:
\begin{itemize}
\item[$-$] Regiao (REG): variável nominal com dois niveis\\
\qquad Regiao A \\
\qquad Região B
\item [$-$] Tempo de entrega (TEE), por encomenda: Variável expressa em u.t.
\item [$-$] Número de encomendas entregues (NEE) por u.t.
\end{itemize}
Admitindo que a amostra disponível é uma amostra aleatória representativa das populações.\\
\\
Neste relatorio esta-se a trabalhar com duas  grandezas precisamente o tempo (TEE) e quantidade por u.t (NEE), temos recolhidos 120 registos \textbf{TEE} na qual pela regra de sturges $c = int(1+3.3log(n))$, determina-se que é necesario sete [\textbf{7}] classes. \\
Podemos obter a amplitude de cada classe $h=b-a$ e sua marca $x_i=\frac{a+b}{2}$. \\
\section{O conjunto de dados}\label{dados}
\noindent
$X_{Ai}$- "Variavel aleatoria que representa o tempo de demora na Região \textbf{A} da entrega de uma encomenda Expresso em u.t." \quad i=1,2,3, .....,117 \\
\qquad$X_{Bi}$- "Variavel aleatoria que representa o tempo de demora na Região \textbf{B} da entrega de uma encomenda Expresso em u.t." \quad i=1,2,3, .....,120 \\
Abaixo o resultado da tabela TEE:\\
\\
\begin{minipage}{0pt}
\begin{tabular}{ |c|c|c|c|c|c|c|c|c|c|c| }
\hline
$h_i$ & CLASSE & MARCA & $nA_i$ & $nB_i$ & $\frac{nA_i}{h_i}$	 & $\frac{nB_i}{h_i}$ & $f_i(A)$	& $f_i(B)$ & $F_i(A)$ & $F_i(B)$ \\
\hline
4 & [5,10[ & 7,5 & 5 & 0 & 1,25 & 0 & 0,0427 & 0 & 0,0427 & 0 \\
\hline
4 & [10,15[ & 12,5 & 16 & 19 & 4 & 4,75 & 0,1368 & 0,1583 & 0,1795 & 0,1583 \\
\hline
4 & [15,20[ & 17,5 & 40 & 28 & 10 & 7 & 0,3419 & 0,2333 & 0,5214 & 0,3917 \\
\hline
4 & [20,25[ & 22,5 & 25 & 41 & 6,25 & 10,25 & 0,2137 & 0,3417 & 0,7350 & 0,7333 \\
\hline
4 & [25,30[ & 27,5 & 26 & 22 & 6,5 & 5,5 & 0,2222 & 0,1833 & 0,9573 & 0,9167 \\
\hline
4 & [30,35[ & 32,5 & 5 & 8 & 1,25 & 2 & 0,0427 & 0,0667 & 1 & 0,9833 \\
\hline
5 & [35,40] & 37,5 & 0 & 2 & 0 & 0,4 & 0 & 0,0167 & 1 & 1 \\
\hline
& & & n=117 & n=120 & & & & & & \\
\hline
\end{tabular}
\end{minipage}
\\
\\
\\
Recorrendo ao excell obeteve-se os seguintes resultados: \\
\begin{minipage}{0pt}
$$\begin{array}{l | l}
\text{Média aritmetica dados classificados} & \text{Variância de uma amostra dados classificados} \\
\overline{x} = \frac{1}{n}\sum_{i=1}^cx_in_i = \sum_{i=1}^cx_if_i & s^2 = \frac{1}{n-1}\sum_{i=1}^c (x_i-\bar{x})^2 n_i
\end{array}$$
\end{minipage}
\\
\begin{minipage}[!b]{0.40\linewidth}
\begin{tabular}{ l c c }
\hline
Estatística & $X_A$ & $X_B$ \\
\hline
Mínimo & 7,5 & 12,5\\
$Q_1$:$1^o$ Quartil & 17,5 & 17,5 \\
$m_d$: mediana & 17,5 & 22,5\\
$Q_3$:$3^o$ Quartil & 27,5 & 27,5 \\
Máximo & 32,5 & 37,5 \\
\hline
$\bar{X}$ : Média & 20,3205 & 21,5833 \\
$s$ : desvio-padrão & 6,1020 & 6,0106\\
$m_o$: moda & 17,5 & 22,5\\
\hline
Tamanho amostral [$n$] & 117 & 120 \\
\hline
\end{tabular}
\label{Tab:Resulatdos}
\end{minipage}
\hspace{2cm}
\begin{minipage}[!b]{0.40\linewidth}
\begin{figure}[H]
\centering
\includegraphics[scale=0.5]{./image/TEE.png}
\caption{TEE}
\label{TEE}
\end{figure}
\end{minipage}
\newpage
\noindent
Na Região \textbf{A} a Média > Moda = Mediana \\ 
com skew = -0,02876 e kurt = -0,50909 \\
Na Região \textbf{B} a Média < Moda = Mediana \\
com skew = 0,14205 e kurt = -212187 \\

\noindent
Na prática, considera-se que a qualidade da aproximação é suficientemente boa quando $n \geqslant 30$.  \\
Pode-se tomar que $\delta \cong s$. \\
\begin{minipage}[l]{0pt}
$$\left\lbrace\begin{array}{c}
\mu \\
\delta \\
\end{array}\right.$$
\end{minipage}
\hspace{3cm} $\Longrightarrow$ \hspace{3cm}
\begin{minipage}[l]{0pt}
\[\bar{X}=\frac{\sum_{i=1}^nX_i}{n}\sim N \big(\mu;\frac{\delta^2}{n}\big)\]
\end{minipage}\\
\\
$\bar{x}_{A_0}$ = 20,3205 \qquad $\bar{x}_{B_0}$ = 21,5833 \\
$\delta_A$ = 6,1020 \qquad $\delta_B$ = 6,0106\\
\newpage
%%%%%%%%%%%%%%%%%%%%%%%%%%%%%%%%%%%%%%%%%%%%%%%%%%%%%%%%%%%%%%%%%%%%%%%%%%%%%%%%%%%%
\noindent
Tratamento dos dados da Segunda Variavel Aleatótia \\
$Y_{Ai}$- "Variavel aleatoria que representa o numero de encomendas entregues pela Expresso na Regiao \textbf{A} por u.t." \quad i=1,2,3, .....,90 \\
$Y_{Bi}$- "Variavel aleatoria que representa a numero de encomendas entregues pela Expresso na Regiao \textbf{B} por u.t." \quad i=1,2,3, .....,90 \\
Abaixo o resultado da tabela NEE:\\
\\
\begin{minipage}{0pt}
\begin{tabular}{ |c|c|c|c|c|c|c| }
\hline
$Y_i$ & $nA_i$ & $nB_i$ & $f_i(A)$	& $f_i(B)$ & $F_i(A)$ & $F_i(B)$ \\
\hline
3 & 6 & 3 & 0,0667 & 0,0333 & 0,0667 & 0,0333 \\
\hline
4 & 8 & 6 & 0,0889 & 0,0667 & 0,1556 & 0,1 \\
\hline
5 & 19 & 13 & 0,2111 & 0,1444 & 0,3677 & 0,2444 \\
\hline
6 & 15 & 7 & 0,1667 & 0,0778 & 0,5333 & 0,3222 \\
\hline
7 & 13 & 19 & 0,1444 & 0,2111 & 0,6778 & 0,5333 \\
\hline
8 & 11 & 15 & 0,1222 & 0,1667 & 0,8 & 0,7 \\
\hline
9 & 6 & 8 & 0,0667 & 0,0889 & 0,8667 & 0,7889 \\
\hline
10 & 5 & 11 & 0,0556 & 0,1222 & 0,9222 & 0,9111 \\
\hline
11 & 4 & 3 & 0,0444 & 0,0333 & 0,9667 & 0,9444 \\
\hline
12 & 0 & 2 & 0 & 0,0222 & 0,9667 & 0,9667 \\
\hline
13 & 2 & 1 & 0,0222 & 0,0111 & 0,9889 & 0.9778 \\
\hline
14 & 1 & 0 & 0,0111 & 0 & 1 & 0,9778 \\
\hline
15 & 0 & 1 & 0 & 0,0111 & 1 & 0,9889 \\
\hline
16 & 0 & 1 & 0 & 0,0111 & 1 & 1 \\
\hline
\end{tabular}
\end{minipage}
\\
\begin{minipage}[!b]{0.40\linewidth}
\begin{tabular}{ l c c }
\hline
Estatística & $Y_A$ & $Y_B$ \\
\hline
Mínimo & 3 & 3 \\
$Q_1$:$1^o$ Quartil & 5 & 6 \\
$m_d$: mediana & 6 & 7 \\
$Q_3$:$3^o$ Quartil & 8 & 9 \\
Máximo & 14 & 16 \\
\hline
$\bar{Y}$ : Média & 6,6889 & 7,5111 \\
$s$ : desvio-padrão & 2,4062 & 2,5139\\
$m_o$: moda & 5 & 7\\
\hline
Tamanho amostral [$n$] & 90 & 90 \\
\hline
\end{tabular}
\label{Tab:Resulatdos}
\end{minipage}
\hspace{1cm}
\begin{minipage}[!b]{0.40\linewidth}
\begin{figure}[H]
\centering
\includegraphics[scale=0.5]{./image/NEE.png}
\caption{NEE}
\label{NEE}
\end{figure}
\end{minipage}
\\
\\
\begin{minipage}[l]{0pt}
$$\left\lbrace\begin{array}{c}
\mu \\
\delta \\
\end{array}\right.$$
\end{minipage}
\hspace{3cm} $\Longrightarrow$ \hspace{3cm}
\begin{minipage}[l]{0pt}
\[\bar{Y}=\frac{\sum_{i=1}^nY_i}{n}\sim N \big(\mu;\frac{\delta^2}{n}\big)\]
\end{minipage}\\
\\
$\bar{y}_{A_0}$ = 6,6889 \qquad $\bar{y}_{B_0}$ = 7,5111 \\
$\delta_A$ = 2,4062 \qquad $\delta_B$ = 2,5139 \\











%%%%%%%%%%%%%%%%%%%%%%%%%%%%%%%%%%%%%%%%%%%%%%%%%%%%%%%%%%%%%%%%%%%%%%%%%%%%%%%%%%%%%
\newpage
\section{Metodologia Estatística}\label{Metodos}
\noindent
\subsection{$TEE \quad IC_{95\%}$}
Estimação do tempo médio para as regiões \textbf{A} e \textbf{B} com um indice de confiança de 95\%. \\
\\
$IC_{1-\alpha}=\left[ A, B\right]$ ; para $1-\alpha = 0.95$, $\alpha=0.05$, $\frac{\alpha}{2}=0.025$ \\
Zona critica $Z_c=\Phi^{-1}(0.025)\approx1.96$ \\
$P\left( A \leqslant \mu \leqslant B \right) = 1-\alpha$ \\
$\triangle=Z_c\times\frac{\delta}{\sqrt{n}}$ \\
$A = \bar{x}-\triangle \qquad and \qquad B = \bar{x}+\triangle$ \\
$\therefore$\\
$IC_{A_{0.95}}=\left[ \quad 19.2148 \: , \: 21.4262 \quad \right]$ \hspace{1cm} and \hspace{1cm} $IC_{B_{0.95}}=\left[ \quad 20.5078 \: , \: 22.6587 \quad \right]$\\
\\
\noindent
Pode-se estimar que o tempo médio $\left[ \; \mu \; \right]$ de entrega na população  esta dentro dos intervalos acima mencionados com 95\% de confiança.\\
\subsection{$NEE$}
\noindent
Verificar se os dados permitem afirmar que existe diferença significativa entre a \% de períodos com menos de \textbf{6} entregas por u.t. na região \textbf{A} e na região \textbf{B}. Responda com base num intervalo de confiança de 97\%. \\
P($X_A$ < 6)= ? \quad e \quad P($X_B$ < 6)= ? \quad then \quad $P_A-P_B$ \quad then \quad $IC_{97\%}$






\section{Resultados e interpretação}\label{Resultados}
fazer tabela só com resultados\\
\section{Conclusões}\label{Conclusão}


%%%%%%%%%%%%%%%%%%%%%%%%%%%%%%%%%%%%%%%%%%%%%%%%%%%%%%%%%%%%%%%%%%%%%%%%%%%%%%%%%%%%%
%\input{./input/EQUACAO}
%\input{./input/DEFINICAO}
%\newpage
%\input{Equacoes}
%%%%%%%%%%%%%%%%%%%%%%%%%%%%%%%%%%%%%%%%%%%%%%%%%%%%%%%%%%%%%%%%%%%%%%%%%%%%%%%%%%%%%%
%Figuras Bibliografia Index
\listoffigures
\cite{*}
\bibliography{./bibliography/Bibliography}
%outro metodo mas manual\input{Bibliografia}
%\printindex
\footnote{Apontamentos Estatistica}
\end{document}
%%%%%%%%%%%%%%%%%%%%%%%%%%%%%%%%%%%%%%%%%%%%%%%%%%%%%%%%%%%%%%%%%%%%%%%%%%%%%%%%%%%%%%
