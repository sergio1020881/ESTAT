\input{./input/PREAMBLE}
%
\include{./include/CAPA}
%
\tableofcontents
%
\appendix
%
\pagestyle{plain}%plain headings empty
%%%%%%%%%%%%%%%%%%%%%%%%%%%%%%%%%%%%%%%%%%%%
\newpage
\section{Resumo}\label{Resumo}


Este trabalho consiste no estudo de Estatistica das Entregas Expresso em  duas regioes, as variaveis em estudo é o tempo de demora das entregas e também quantas entregas são enregues num determinado tempo ou unidade de tempo [u.t,]. Na primeira situação foram retiradas 120 amostras e 90 nas quantidades entregues num intervalo de tempo [u.t].

Pretende-se fazer um estudo destas duas variaveis aleatorias de forma a responder a questoes empostas e tomar decisoes, recorresndo aos metodos expostos.

Como se trata de amostras recoremos a Inferencia de forma a tirar conclusoes com um grau de incerteza.

Durante este relatorio sera demonstrado resultados e suas interpretações.

No fim apresentado as conclusões que podemos ou sabemos retirar.





Inferencia - IC é das amostras poder inferir characteristicas as populações com um certo grau de incerteza pre-determinado.


\newpage
%
\section{Introdução}\label{Introdução}
%
Considere o ficheiro “ESTAT1920 TP POR.xlsx” contendo os resultados obtidos num estudo que
pretende analisar o efeito do fator região num sistema de entregas expresso. Foram analisados
os registos de 120 entregas, escolhidas aleatoriamente dentro de cada região, e medido o tempo
de entrega, em unidades de tempo (u.t.). Foram também analisados 90 períodos de tempo de 1
u.t., escolhidos aleatoriamente, e contabilizado o número de entregas de encomendas nas
regiões A e B em cada um desses períodos.\\
\\
As variáveis consideradas foram:
\begin{itemize}
\item[$-$] Regiao (REG): variavel nominal com dois niveis\\
\qquad Regiao A \\
\qquad Região B
\item [$-$] Tempo de entrega (TEE), por encomenda: Variável expressa em u.t.
\item [$-$] Número de encomendas entregues (NEE) por u.t.
\end{itemize}
Admitindo que a amostra disponível é uma amostra aleatória representativa das populações em
estudo, inclua no relatório a resposta aos seguintes pontos:\\
\\
Neste relatorio esta-se a trabalhar com duas  grandezas precisamente o tempo (TEE) e quantidade por u.t (NEE), temos recolhidos 120 regsitos na qual pela regra de sturges $c = int(1+3.3log(n))$, determina-se que é necesario sete [7] Classes.
Tambem podemos obter a amplitude de cada classe $h=b-a$ e sua marcar $x_i=\frac{a+b}{2}$ \\
Na prática, considera-se que a qualidade da aproximação é suficientemente boa quando nmaior ou igua 30 \\
grandes amostras, se n maior ou igaula 30 \\
\\
\section{O conjunto de dados}\label{dados}
Abaixo o resultado da tabela TEE:
\\
\\
\begin{minipage}{0pt}
	\begin{tabular}{ |c|c|c|c|c|c|c|c|c|c|c| }
\hline
$h_i$ & CLASSE & MARCA & $nA_i$ & $nB_i$ & $\frac{nA_i}{h_i}$	 & $\frac{nB_i}{h_i}$ & $f_i(A)$	& $f_i(B)$ & $F_i(A)$ & $F_i(B)$ \\
\hline
4 & [5,10[ & 7,5 & 5 & 0 & 1,25 & 0 & 0,0427 & 0 & 0,0427 & 0 \\
\hline
4 & [10,15[ & 12,5 & 16 & 19 & 4 & 4,75 & 0,1368 & 0,1583 & 0,1795 & 0,1583 \\
\hline
4 & [15,20[ & 17,5 & 40 & 28 & 10 & 7 & 0,3419 & 0,2333 & 0,5214 & 0,3917 \\
\hline
4 & [20,25[ & 22,5 & 25 & 41 & 6,25 & 10,25 & 0,2137 & 0,3417 & 0,7350 & 0,7333 \\
\hline
4 & [25,30[ & 27,5 & 26 & 22 & 6,5 & 5,5 & 0,2222 & 0,1833 & 0,9573 & 0,9167 \\
\hline
4 & [30,35[ & 32,5 & 5 & 8 & 1,25 & 2 & 0,0427 & 0,0667 & 1 & 0,9833 \\
\hline
5 & [35,40] & 37,5 & 0 & 2 & 0 & 0,4 & 0 & 0,0167 & 1 & 1 \\
\hline
& & & n=117 & n=120 & & & & & & \\
\hline
	\end{tabular}
\end{minipage}
\\
\\
Média aritmetica $\bar{x}$ dados classificados \\
\begin{minipage}{0pt}
\[\overline{x} = \frac{1}{n}\sum_{i=1}^cx_in_i = \sum_{i=1}^cx_if_i\]
\end{minipage}
\\
\\
Recorrendo ao excell obeteve-se os seguintes resultados:
\\
\\
$\bar{x_A}=20,3205$ \\
$\bar{x_B}=25,5833$ \\
Variância de uma amostra $s^2$: \\
dados classificados \\
\begin{minipage}{0pt}
\[s^2 = \frac{1}{n-1}\sum_{i=1}^c (x_i-\bar{x})^2 n_i\]
\end{minipage}
\\
\\
$s_A^2=$ \\
$s_B^2=$ \\
\\

\[\bar{X}=\frac{\sum_{i=1}^nX_i}{n}~\big(\mu;\frac{\delta^2}{n}\big)\]



\begin{figure}[H]
\centering
\includegraphics[scale=0.8]{./image/first.png}
\caption{TEE}
\label{TEE}
\end{figure}\par



\section{Metodologia Estatística}\label{Metodos}
\section{Resultados e interpretação}\label{Resultados}
fazer tabela só com resultados\\
\section{Conclusões}\label{Conclusão}


%%%%%%%%%%%%%%%%%%%%%%%%%%%%%%%%%%%%%%%%%%%%%%%%%%%%%%%%%%%%%%%%%%%%%%%%%%%%%%%%%%%%%
%\input{./input/EQUACAO}
%\input{./input/DEFINICAO}
%\newpage
%\input{Equacoes}
%%%%%%%%%%%%%%%%%%%%%%%%%%%%%%%%%%%%%%%%%%%%%%%%%%%%%%%%%%%%%%%%%%%%%%%%%%%%%%%%%%%%%%
%Figuras Bibliografia Index
\listoffigures
\cite{*}
\bibliography{./bibliography/Bibliography}
%outro metodo mas manual\input{Bibliografia}
%\printindex
\footnote{Apontamentos Estatistica}
\end{document}
%%%%%%%%%%%%%%%%%%%%%%%%%%%%%%%%%%%%%%%%%%%%%%%%%%%%%%%%%%%%%%%%%%%%%%%%%%%%%%%%%%%%%%
%Variância de uma amostra $s^2$: \\
%dados nao clasificados \\
%\begin{minipage}{0pt}
%\begin{flalign*}
%s^2 = \frac{1}{n-1}\sum_{i=1}^n (x_i-\bar{x})^2
%\end{flalign*}
%\end{minipage}
%\\
%\\
%dados classificados
%\[s^2 = \frac{1}{n-1}\sum_{i=1}^c (x_i-\bar{x})^2 n_i\]
%Desvio padrao\\
%$s = \sqrt{s^2}$
%coeficiente de variacao\\
%\[cv = \frac{s}{\bar{x}}\]
%frequência relativa\\
%\[f_i = \frac{n_i}{n}\]
%\[\sum_{i=1}^cn_i = n\]
%\[\sum_{1=1}^cf_i = 1\]
%frequência acumulada\\
%\[N_i = \sum_{j=1}^in_j\]
%frequencia relativa acumulada\\
%\[F_i = \sum_{j=1}^if_i\]
%Média aritmetica \[\bar{x}\] dados nao classificados\\
%\[\bar{x} = \frac{1}{n}\sum_{1=1}^n\]
%Média aritmetica \[\bar{x}\] dados classificados\\
%\[\bar{x} = \frac{1}{n}\sum_{i=1}^cx_in_i = \sum_{i=1}^cx_if_i\]
%Mediana para dados não classificados em série ordenada
%\[ M_e= 
%\begin{cases}
%    \frac{X_{n/2}+X_{n/2+1}}{2}, & n \quad par \\
%    \frac{X_{n+1}}{2} & n \quad impar
%\end{cases}
%\]
%Mediana para dados quantitativos continuos classificados
%\[ M_e=
%L_k+\left(\frac{0.5-F_{k-1}}{f_k}\right)h_k
%\]
%Quartil para dados não classificados em série ordenada
%\[z_\alpha=x_k\]
%onde $x_k$ é o maior inteiro menor que $n_{\alpha+1}$
%Quartil para dados quantitativos contínuos classificados
%\[z_\alpha=L_k+\left(\frac{\alpha-F_{k-1}}{f_k}\right)h_k\]
%Momento centrado de ordem r,$m_r$
%Para dados não classificados:
%\[m_r=\frac{1}{n}\sum_{i=1}^n(x_i-\bar{x})^r\] 
%Para dados classificados:
%\[m_r=\frac{1}{n}\sum_{i=1}^c(x_i-\bar{x})^rn_i\]
%Coeficiente de assimetria amostral, $a_3$
%\[a_3=\frac{m_3}{s^3}\]
%Coeficiente de curtose amostral, $a_4$
%\[a_4=\frac{m_4}{s^4}\]
%
%\end{equation}
