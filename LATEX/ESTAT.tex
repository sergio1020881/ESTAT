\input{./input/PREAMBLE}
%
\include{./include/CAPA}
%
\tableofcontents
%
\appendix
%
\pagestyle{plain}%plain headings empty
%%%%%%%%%%%%%%%%%%%%%%%%%%%%%%%%%%%%%%%%%%%%
\newpage
\label{Resumo}
\begin{abstract}
Este trabalho consiste no estudo de Estatistica das Entregas Expresso em  duas regioes \textbf{A} e \textbf{B}, as variaveis em estudo é o tempo de demora das entregas e a variavel de numero de encomendas entregues num determado unidade de tempo [u.t.]. Nestas situações foram retiradas 120 e 90 amostras nas duas regiões respectivamente.\\
A primeira é uma distribuição continua, o tempo, e a segunda uma distribuição discreta.
\\
As materias abordadas vai ser \textbf{Amostragem}, \textbf{Estimação de parâmetros} e \textbf{Testes de Hipóteses}\\
\end{abstract}
%
\newpage
\section{Introdução}\label{Introdução}
%
As variáveis consideradas foram:
\begin{itemize}
\item[$-$] Regiao (REG): variavel nominal com dois niveis\\
\qquad Regiao A \\
\qquad Região B
\item [$-$] Tempo de entrega (TEE), por encomenda: Variável expressa em u.t.
\item [$-$] Número de encomendas entregues (NEE) por u.t.
\end{itemize}
Admitindo que a amostra disponível é uma amostra aleatória representativa das populações.\\
\\
Neste relatorio esta-se a trabalhar com duas  grandezas precisamente o tempo (TEE) e quantidade por u.t (NEE), temos recolhidos 120 registos \textbf{TEE} na qual pela regra de sturges $c = int(1+3.3log(n))$, determina-se que é necesario sete [\textbf{7}] classes. \\
Podemos obter a amplitude de cada classe $h=b-a$ e sua marcar $x_i=\frac{a+b}{2}$. \\
\section{O conjunto de dados}\label{dados}
\noindent
$X_{Ai}$- "Variavel aleatoria que representa o tempo de demora na Regiao \textbf{A} da entrega de uma encomenda Expresso em u.t." \quad i=1,2,3, .....,117 \\
\qquad$X_{Bi}$- "Variavel aleatoria que representa o tempo de demora na Regiao \textbf{B} da entrega de uma encomenda Expresso em u.t." \quad i=1,2,3, .....,120 \\
Abaixo o resultado da tabela TEE:\\
\\
\begin{minipage}{0pt}
\begin{tabular}{ |c|c|c|c|c|c|c|c|c|c|c| }
\hline
$h_i$ & CLASSE & MARCA & $nA_i$ & $nB_i$ & $\frac{nA_i}{h_i}$	 & $\frac{nB_i}{h_i}$ & $f_i(A)$	& $f_i(B)$ & $F_i(A)$ & $F_i(B)$ \\
\hline
4 & [5,10[ & 7,5 & 5 & 0 & 1,25 & 0 & 0,0427 & 0 & 0,0427 & 0 \\
\hline
4 & [10,15[ & 12,5 & 16 & 19 & 4 & 4,75 & 0,1368 & 0,1583 & 0,1795 & 0,1583 \\
\hline
4 & [15,20[ & 17,5 & 40 & 28 & 10 & 7 & 0,3419 & 0,2333 & 0,5214 & 0,3917 \\
\hline
4 & [20,25[ & 22,5 & 25 & 41 & 6,25 & 10,25 & 0,2137 & 0,3417 & 0,7350 & 0,7333 \\
\hline
4 & [25,30[ & 27,5 & 26 & 22 & 6,5 & 5,5 & 0,2222 & 0,1833 & 0,9573 & 0,9167 \\
\hline
4 & [30,35[ & 32,5 & 5 & 8 & 1,25 & 2 & 0,0427 & 0,0667 & 1 & 0,9833 \\
\hline
5 & [35,40] & 37,5 & 0 & 2 & 0 & 0,4 & 0 & 0,0167 & 1 & 1 \\
\hline
& & & n=117 & n=120 & & & & & & \\
\hline
\end{tabular}
\end{minipage}
\\
\\
\\
Recorrendo ao excell obeteve-se os seguintes resultados: \\
\begin{minipage}{0pt}
$$\begin{array}{l | l}
\text{Média aritmetica dados classificados} & \text{Variância de uma amostra dados classificados} \\
\overline{x} = \frac{1}{n}\sum_{i=1}^cx_in_i = \sum_{i=1}^cx_if_i & s^2 = \frac{1}{n-1}\sum_{i=1}^c (x_i-\bar{x})^2 n_i
\end{array}$$
\end{minipage}
\\
\begin{minipage}[!b]{0.40\linewidth}
\begin{tabular}{ l c c }
\hline
Estatística & $X_A$ & $X_B$ \\
\hline
Mínimo & 7,5 & 12,5\\
$Q_1$:$1^o$ Quartil & 17,5 & 17,5 \\
$m_d$: mediana & 17,5 & 22,5\\
$Q_3$:$3^o$ Quartil & 27,5 & 27,5 \\
Máximo & 32,5 & 37,5 \\
\hline
$\bar{X}$ : Média & 20,3205 & 21,5833 \\
$s$ : desvio-padrão & 6,1020 & 6,0106\\
$m_o$: moda & 17,5 & 22,5\\
\hline
Tamanho amostral & 117 & 120 \\
\hline
\end{tabular}
\label{Tab:Resulatdos}
\end{minipage}
\hspace{2cm}
\begin{minipage}[!b]{0.40\linewidth}
\begin{figure}[H]
\centering
\includegraphics[scale=0.5]{./image/TEE.png}
\caption{TEE}
\label{TEE}
\end{figure}
\end{minipage}
\newpage
\noindent
Na Região \textbf{A} a Média > Moda = Mediana \\ 
com skew = -0,02876 e kurt = -0,50909 \\
Na Região \textbf{B} a Média < Moda = Mediana \\
com skew = 0,14205 e kurt = -212187 \\

\noindent
Na prática, considera-se que a qualidade da aproximação é suficientemente boa quando $n \geqslant 30$.  \\
Pode-se tomar que $\delta \cong s$. \\
%\right{asasas}


%
\begin{minipage}{0pt}
\[\bar{X}=\frac{\sum_{i=1}^nX_i}{n}\sim N \big(\mu;\frac{\delta^2}{n}\big)\]
\end{minipage}\\
\newpage
%%%%%%%%%%%%%%%%%%%%%%%%%%%%%%%%%%%%%%%%%%%%%%%%%%%%%%%%%%%%%%%%%%%%%%%%%%%%%%%%%%%%
\noindent
Tratamento dos dados da Segunda Variavel Aleatótia \\
$Y_{Ai}$- "Variavel aleatoria que representa o numero de encomendas entregues pela Expresso na Regiao \textbf{A} por u.t." \quad i=1,2,3, .....,90 \\
$Y_{Bi}$- "Variavel aleatoria que representa a numero de encomendas entregues pela Expresso na Regiao \textbf{B} por u.t." \quad i=1,2,3, .....,90 \\
Abaixo o resultado da tabela NEE:\\
\\
\begin{minipage}{0pt}
\begin{tabular}{ |c|c|c|c|c|c|c| }
\hline
Enc/u.t & $nA_i$ & $nB_i$ & $f_i(A)$	& $f_i(B)$ & $F_i(A)$ & $F_i(B)$ \\
\hline
3 & 6 & 3 & 0,0667 & 0,0333 & 0,0667 & 0,0333 \\
\hline
4 & 8 & 6 & 0,0889 & 0,0667 & 0,1556 & 0,1 \\
\hline
5 & 19 & 13 & 0,2111 & 0,1444 & 0,3677 & 0,2444 \\
\hline
6 & 15 & 7 & 0,1667 & 0,0778 & 0,5333 & 0,3222 \\
\hline
7 & 13 & 19 & 0,1444 & 0,2111 & 0,6778 & 0,5333 \\
\hline
8 & 11 & 15 & 0,1222 & 0,1667 & 0,8 & 0,7 \\
\hline
9 & 6 & 8 & 0,0667 & 0,0889 & 0,8667 & 0,7889 \\
\hline
10 & 5 & 11 & 0,0556 & 0,1222 & 0,9222 & 0,9111 \\
\hline
11 & 4 & 3 & 0,0444 & 0,0333 & 0,9667 & 0,9444 \\
\hline
12 & 0 & 2 & 0 & 0,0222 & 0,9667 & 0,9667 \\
\hline
13 & 2 & 1 & 0,0222 & 0,0111 & 0,9889 & 0.9778 \\
\hline
14 & 1 & 0 & 0,0111 & 0 & 1 & 0,9778 \\
\hline
15 & 0 & 1 & 0 & 0,0111 & 1 & 0,9889 \\
\hline
16 & 0 & 1 & 0 & 0,0111 & 1 & 1 \\
\hline
\end{tabular}
\end{minipage}
\\
\begin{minipage}[!b]{0.40\linewidth}
\begin{tabular}{ l c c }
\hline
Estatística & $X_A$ & $X_B$ \\
\hline
Mínimo & 3 & 3 \\
$Q_1$:$1^o$ Quartil & 5 & 6 \\
$m_d$: mediana & 6 & 7 \\
$Q_3$:$3^o$ Quartil & 8 & 9 \\
Máximo & 14 & 16 \\
\hline
$\bar{X}$ : Média & 6,6889 & 7,5111 \\
$s$ : desvio-padrão & 2,4062 & 2,5139\\
$m_o$: moda & 5 & 7\\
\hline
Tamanho amostral & 90 & 90 \\
\hline
\end{tabular}
\label{Tab:Resulatdos}
\end{minipage}
\hspace{1cm}
\begin{minipage}[!b]{0.40\linewidth}
\begin{figure}[H]
\centering
\includegraphics[scale=0.5]{./image/NEE.png}
\caption{NEE}
\label{NEE}
\end{figure}
\end{minipage}












%%%%%%%%%%%%%%%%%%%%%%%%%%%%%%%%%%%%%%%%%%%%%%%%%%%%%%%%%%%%%%%%%%%%%%%%%%%%%%%%%%%%%
\newpage






\section{Metodologia Estatística}\label{Metodos}
\section{Resultados e interpretação}\label{Resultados}
fazer tabela só com resultados\\
\section{Conclusões}\label{Conclusão}


%%%%%%%%%%%%%%%%%%%%%%%%%%%%%%%%%%%%%%%%%%%%%%%%%%%%%%%%%%%%%%%%%%%%%%%%%%%%%%%%%%%%%
%\input{./input/EQUACAO}
%\input{./input/DEFINICAO}
%\newpage
%\input{Equacoes}
%%%%%%%%%%%%%%%%%%%%%%%%%%%%%%%%%%%%%%%%%%%%%%%%%%%%%%%%%%%%%%%%%%%%%%%%%%%%%%%%%%%%%%
%Figuras Bibliografia Index
\listoffigures
\cite{*}
\bibliography{./bibliography/Bibliography}
%outro metodo mas manual\input{Bibliografia}
%\printindex
\footnote{Apontamentos Estatistica}
\end{document}
%%%%%%%%%%%%%%%%%%%%%%%%%%%%%%%%%%%%%%%%%%%%%%%%%%%%%%%%%%%%%%%%%%%%%%%%%%%%%%%%%%%%%%
%\\
%Desvio padrao\\
%$s = \sqrt{s^2}$
%coeficiente de variacao\\
%\[cv = \frac{s}{\bar{x}}\]
%Mediana para dados quantitativos continuos classificados
%\[ M_e=
%L_k+\left(\frac{0.5-F_{k-1}}{f_k}\right)h_k
%\]
%Quartil para dados quantitativos contínuos classificados
%\[z_\alpha=L_k+\left(\frac{\alpha-F_{k-1}}{f_k}\right)h_k\]
%Momento centrado de ordem r,$m_r$
%Para dados classificados:
%\[m_r=\frac{1}{n}\sum_{i=1}^c(x_i-\bar{x})^rn_i\]
%Coeficiente de assimetria amostral, $a_3$
%\[a_3=\frac{m_3}{s^3}\]
%Coeficiente de curtose amostral, $a_4$
%\[a_4=\frac{m_4}{s^4}\]
%frequência relativa\\
%\[f_i = \frac{n_i}{n}\]
%\[\sum_{i=1}^cn_i = n\]
%\[\sum_{1=1}^cf_i = 1\]
%frequência acumulada\\
%\[N_i = \sum_{j=1}^in_j\]
%frequencia relativa acumulada\\
%\[F_i = \sum_{j=1}^if_i\]
%Inferencia - IC é das amostras poder inferir characteristicas as populações com um certo grau de incerteza pre-determinado.
%Mediana para dados não classificados em série ordenada
%\[ M_e= 
%\begin{cases}
%    \frac{X_{n/2}+X_{n/2+1}}{2}, & n \quad par \\
%    \frac{X_{n+1}}{2} & n \quad impar
%\end{cases}
%\]
%Quartil para dados não classificados em série ordenada
%\[z_\alpha=x_k\]
%onde $x_k$ é o maior inteiro menor que $n_{\alpha+1}$
%Média aritmetica \[\bar{x}\] dados nao classificados\\
%\[\bar{x} = \frac{1}{n}\sum_{1=1}^n\]
%Variância de uma amostra $s^2$: \\
%dados nao clasificados \\
%\begin{minipage}{0pt}
%\begin{flalign*}
%s^2 = \frac{1}{n-1}\sum_{i=1}^n (x_i-\bar{x})^2
%\end{flalign*}
%\end{minipage}
%Para dados não classificados:
%\[m_r=\frac{1}{n}\sum_{i=1}^n(x_i-\bar{x})^r\]
%Como se trata de amostras recoremos a Inferencia de forma a tirar conclusoes com um grau de incerteza.\\
%Durante este relatorio sera demonstrado resultados e suas interpretações.\\
%No fim apresentado as conclusões que podemos ou sabemos retirar.\\
%Pretende-se fazer um estudo destas variaveis aleatorias de forma a responder a questoes empostas e tomar decisoes.\\